
\documentclass{beamer}
\usetheme{ucl}

%%% Increase the height of the banner: the argument is a scale factor >=1.0
%\setbeamertemplate{banner}[ucl][0.1]

%%% Change the colour of the main banner
%%% The background should be one of the UCL colours (except pink or white):
%%%   black,darkpurple,darkred,darkblue,darkgreen,darkbrown,richred,midred,
%%%   navyblue,midgreen,darkgrey,orange,brightblue,brightgreen,lightgrey,
%%%   lightpurple,yellow,lightblue,lightgreen,stone
\setbeamercolor{banner}{bg=darkpurple}
%\setbeamercolor{banner}{bg=yellow,fg=black}

%%% Add a stripe behind the banner
%\setbeamercolor{banner stripe}{bg=darkpurple,fg=black}

%%% The main structural elements
\setbeamercolor{structure}{fg=black}

%%% Author/Title/Date and slide number in the footline
\setbeamertemplate{footline}[author title date]

%%% Puts the section/subsection in the headline
% \setbeamertemplate{headline}[section]

%%% Puts a navigation bar on top of the banner
%%% For this to work correctly, the each \section command needs to be
%%% followed by a \subsection. Requires one extra compile.
% \setbeamertemplate{headline}[miniframes]
%%% Accepts an optional argument determining the width
% \setbeamertemplate{headline}[miniframes][0.3\paperwidth]


%%% Puts the frame title in the banner
%%% Won't work correctly with the above headline templates
%\useoutertheme{ucltitlebanner}
%%% Similar to above, but smaller (and puts subtitle on same line as title)
\useoutertheme[small]{ucltitlebanner}

%%% Gives block elements (theorems, examples) a border
% \useinnertheme{blockborder}
%%% Sets the body of block elements to be clear
% \setbeamercolor{block body}{bg=white,fg=black}

%%% Include CSML logo on title slide
%\titlegraphic{\includegraphics[width=0.16\paperwidth]{csml_logo}}

%%% Include CSML logo in bottom right corner of all slides
%\logo{\includegraphics[width=0.12\paperwidth]{csml_logo}}

%%% Set a background colour
% \setbeamercolor{background canvas}{bg=lightgrey}

%%% Set a background image
%%% Some sample images are available from the UCL image store:
%%%   https://www.imagestore.ucl.ac.uk/home/start
% \setbeamertemplate{background canvas}{%
%   \includegraphics[width=\paperwidth]{imagename}}



%%%%%% Some other settings that can make things look nicer
%%% Set a smaller indent for description environment
\setbeamersize{description width=2em}
%%% Remove nav symbols (and shift any logo down to corner)
\setbeamertemplate{navigation symbols}{\vspace{-2ex}}








\DeclareMathOperator{\Cov}{Cov}
\DeclareMathOperator{\Var}{Var}
\DeclareMathOperator{\E}{\mathbb{E}}
\DeclareMathOperator{\Proba}{\mathbb{P}}

\newcommand{\Covb}[2]{\ensuremath{\Cov\!\left[#1,#2\right]}}
\newcommand{\Eb}[1]{\ensuremath{\E\!\left[#1\right]}}
\newcommand{\Pb}[1]{\ensuremath{\Proba\!\left[#1\right]}}
\newcommand{\Varb}[1]{\ensuremath{\Var\!\left[#1\right]}}

% norm
\newcommand{\norm}[1]{\| #1 \|}

\newcommand{\indep}{\rotatebox[origin=c]{90}{$\models$}}





\usepackage{mathptmx,amsmath,amssymb,graphicx,bibentry,bbm,ragged2e}
\usepackage[english]{babel}

\makeatletter

\newcommand{\noun}[1]{\textsc{#1}}
\newcommand{\jitem}[1]{\item \begin{justify} #1 \end{justify} \vfill{}}
\newcommand{\sframe}[2]{\frame{\frametitle{#1} #2}}

\newenvironment{centercolumns}{\begin{columns}[c]}{\end{columns}}
%\newenvironment{jitem}{\begin{justify}\begin{itemize}}{\end{itemize}\end{justify}}



%\usetheme{Warsaw}
%\setbeamertemplate{footline}[text line]{}
%\setbeamertemplate{headline}{}
%\setbeamercolor{structure}{fg=purple!50!blue, bg=purple!50!blue}

%\setbeamersize{text margin left=15pt,text margin right=15pt}

%\setbeamercovered{transparent}


\@ifundefined{showcaptionsetup}{}{%
 \PassOptionsToPackage{caption=false}{subfig}}
\usepackage{subfig}

\usepackage[utf8]{inputenc}
\usepackage[T1]{fontenc}

\usepackage{multirow}


\makeatother

\def \draft {1}

\usepackage{xparse}
\usepackage{ifthen}
\DeclareDocumentCommand{\comment}{m o o o o}
{\ifthenelse{\draft=1}{
    \textcolor{red}{\textbf{C : }#1}
    \IfValueT{#2}{\textcolor{blue}{\textbf{A1 : }#2}}
    \IfValueT{#3}{\textcolor{ForestGreen}{\textbf{A2 : }#3}}
    \IfValueT{#4}{\textcolor{red!50!blue}{\textbf{A3 : }#4}}
    \IfValueT{#5}{\textcolor{Aquamarine}{\textbf{A4 : }#5}}
 }{}
}
\newcommand{\todo}[1]{
\ifthenelse{\draft=1}{\textcolor{red!50!blue}{\textbf{TODO : \textit{#1}}}}{}
}




\begin{document}

\title[A Critique of Urban Genome Project]{A Critique of Urban Genome Project}
\author[Raimbault]{J.~Raimbault$^{1,2,3\ast}$\\\medskip
$^{\ast}$\texttt{j.raimbault@ucl.ac.uk}
}

\institute[UCL]{$^{1}$Center for Advanced Spatial Analysis, University College London\\
$^{2}$UPS CNRS 3611 Complex Systems Institute Paris\\
$^{3}$UMR CNRS 8504 G{\'e}ographie-cit{\'e}s
}


\date[July 18th 2020]{Urban Genome Project seminar\\
University of Toronto, School of Cities\\
December 9th 2020
}

\frame{\maketitle}

\AtBeginSection[]
{
	\frame{
		\tableofcontents[currentsection, hideallsubsections]
	}
	\addtocounter{framenumber}{-1}
}


%

\section{General comments}



\sframe{Modeling urban evolution}{

% from social evol to urban evol?

}


\section{Specific comments}

% Linear reading

\subsection{Part I}

\subsubsection{Introduction}

% - cities major settlement form? cf Mike: before cities - after? - transition?
% - cit new Science of cities: not sure Mike meant in such an integrative way - cf preprint Elsa?
% - no consensus: better this way? cd urban perspective. Can it be a true unified explanatory model?
% beg. p4: theories and models?
% p5: concept vs variable - what exactly is a variable here?
% Q: can it be "everything"? "different form of urban life" => examples with no evolutionary processes?
% on the need of interdisciplinarity: particularly agree - urb persp

\sframe{Part I - Introduction}{

\begin{itemize}
	\item \cite{batty2013new} one view on urban systems; integrative approaches see \cite{lobo2020urban}
	\item Link and distinction between theories and models? see \cite{livet2010ontology} \cite{raimbault2017applied}
	\item Should urban evolution encompass ``everything urban''? $\rightarrow$ examples on the opposite of evolutionary processes?
	\item On the need for interdisciplinarity and multiple perspectives, totally agree: see \cite{pumain2020conclusion}
\end{itemize}


}

\subsubsection{Literature review}

% Urban ecology : // industrial symbiosis?
% Complex adaptive systems: indeed morphogenesis models are not models of urban evolution - confusion in concepts.
% Scaling: allometric growth: do not cite Chen - ! diff in time / static - Q of ergodicity

% bio -> social: units become themselves more complex, heterogenous, fuzzy.

\sframe{Part I - Literature review}{

\begin{itemize}
	\item On urban ecology, see also industrial symbiosis: analogy or equivalence? \cite{raimbault2020spatial}
	\item On Complex Adaptive Systems approach: indeed morphogenesis models are not evolution models - confusion of concepts in the literature? (as in evolutionary economic geography in general? \cite{schamp2010notion})
	\item On Scaling Theories: allometric growth/scaling is linked to scaling but not the same concept - \cite{chen2009analytical} \cite{lang2019reinvestigating}; issue of non-ergodicity \cite{pumain2012multi}
	\item When switching from biological to social evolution, evolution units become more complex, heterogenous, fuzzy.
\end{itemize}


}


\subsubsection{Model outline}

% disagreements analogy or more universal process? -> importance of understanding epistemo aspects of transfers of concepts / universal processes (structural analogy of models/theories?) / nature of interdisciplinarity
% p16: what is a sociocultural entity is fuzzy (at this stage?)

% Proposing physical entities as elementary units is challenging:
%.  - link form and function?
%.  - indeed encode information on how to best organize space, but multi-obj optim by multiple agents (so could not be "simple results of selection"?) -> OK p19 "rational without necessarily being optimized to the interests of any human actors"
%.  - def of elementary unit, where to draw the boundary in terms of nature and delimitation - how to define the system - scales?
%.  - functions / informational processes (cit info in complex systems) may be missed? -> ok function ~ encoded: "groups and activities towards which they are oriented" "a way of physically organizing space for some sets of activities and groups" "Be made out of this stuff arranged in this way, for doing these things for groups of people like this"
%   - Q: def of "socially organized"?
%  - Q of scale: micro level of representation only? - idea of a "molar unit": not sure it is useful/necessary?
%  - levels of interaction and strong emergence "After all, people do not interact with such abstract entities" - they do: admin entities, state, etc.


\sframe{Model outline}{

\begin{itemize}
	\item Evolution as an analogy or as underlying general processes? $\rightarrow$ transfer of concepts/theories/models between disciplines - nature of interdisciplinarity
	\item What exactly is a ``socio-cultural entity'' is still fuzzy at this stage
\end{itemize}

Proposing ``physical'' urban entities is challenging on several points:

\begin{itemize}
	\item Link form-function? (included in definition of formeme)
	\item Selected as optimizing spatial organization - but fitness functions in social/urban environments are tied to multiple actors at multiple scales, with multiple objectives	
	\item How to draw the boundary for units (nature, scale, delimitation), how to define the system
	\item Information essential in complex systems (included in the formeme) \cite{haken2006information}
	\item Why necessarily a ``molar unit''? Can it span multiple scales?
	\item Compatibility with weak/strong emergence? (people interact with higher level entities: state, municipal council) \cite{bedau2000open}
\end{itemize}




}


\subsubsection{Conclusion}

% origin of cities? yes and no -> cf ALife strongly interested in origins of life - these must be evolutionary (prebiotic theories: chemistry evolution) (see comment on Walker, transitions/information)
% reductionism/determinism: totally agree (complexity of evolution)
% not "teleological": should always be for some agents/utility function? (cf Monod \cite{monod2014hasard}
% from the bottom-up - daily life: agree
% mimicry of bio evol? totally agree - cf comment on transfer of concepts between disciplines
% not anti planning: complexity of managing/ designing complex systems from the bottom up: cf field of morphogenetic engineering \cite{doursat2012morphogenetic}

\sframe{Part I: conclusion}{

\begin{itemize}
	\item Origin and cities: 
	\item Not ``teleological''
\end{itemize}

}



\subsection{Part II}


\subsubsection{Introduction}

% signal -> signals and boundaries


\section{Discussion}

% level of complexity/evolution? - successive evolutions - necessaries? // talk ALife Krakauer https://scholar.google.com/citations?user=R-K-FOwAAAAJ&hl=fr&oi=ao : ? - inevitability of life / intelligence
% \cite{cronin2016beyond} Sara Walker  life as a phase transition, life/non-life (biosignatures)
% "without an understanding of what life is, how can we approach understanding its origins" \cite{walker2017re}

% multidimensionality?

% biological metaphors / transfer of concepts: ex symbiosis, morphogenesis: epistemo work on transfer of concepts?

% Ambitious new way of revisiting link between form and function: // disc morphogenesis

% spatio-temporal aspect of evolution / co-evolution niches (cit. Holland?)

% autopoiesis: Q of system def and boundaries

\sframe{Discussion}{


}







%%%%%%%%%%%%%%%%%%%%%
\begin{frame}[allowframebreaks]
\frametitle{References}
\bibliographystyle{apalike}
\bibliography{biblio}
\end{frame}
%%%%%%%%%%%%%%%%%%%%%%%%%%%%










\end{document}
