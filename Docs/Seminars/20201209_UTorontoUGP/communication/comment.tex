\input{header_slides.tex}

\begin{document}

\title[A Critique of Urban Genome Project]{A Critique of Urban Genome Project}
\author[Raimbault]{J.~Raimbault$^{1,2,3\ast}$\\\medskip
$^{\ast}$\texttt{j.raimbault@ucl.ac.uk}
}

\institute[UCL]{$^{1}$Center for Advanced Spatial Analysis, University College London\\
$^{2}$UPS CNRS 3611 Complex Systems Institute Paris\\
$^{3}$UMR CNRS 8504 G{\'e}ographie-cit{\'e}s
}


\date[July 18th 2020]{Urban Genome Project seminar\\
University of Toronto, School of Cities\\
December 9th 2020
}

\frame{\maketitle}

\AtBeginSection[]
{
	\frame{
		\tableofcontents[currentsection, hideallsubsections]
	}
	\addtocounter{framenumber}{-1}
}


%

\section{General comments}



\sframe{General comments}{

% from social evol to urban evol?

\begin{itemize}
	\item Link with geographical/economic  stylized facts at multiple scales seems to be weak - urban theory construction is generally conditioned by available data
	\item Constructing a formal model of urban evolution is not a guarantee of explanation (``answering the why''): the actual processes corresponding to the semantics will capture the causal structure of systems
	\item Definitions of a city, of urban systems, etc. are not constructed in strong link with the theory and with empirical observations
	\item The model formal development, which can be specified in many ways them , is a good thing as it provides a general framework; but as it is now it leaves too much doubt on how to actually measure these processes, the signals, the diffusion of information, etc.
	\item Does not seem to allow endogeneity and open-ended evolution
\end{itemize}




}


\section{Specific comments}

% Linear reading

\subsection{Part I}

\subsubsection{Introduction}

% - cities major settlement form? cf Mike: before cities - after? - transition?
% - cit new Science of cities: not sure Mike meant in such an integrative way - cf preprint Elsa?
% - no consensus: better this way? cd urban perspective. Can it be a true unified explanatory model?
% beg. p4: theories and models?
% p5: concept vs variable - what exactly is a variable here?
% Q: can it be "everything"? "different form of urban life" => examples with no evolutionary processes?
% on the need of interdisciplinarity: particularly agree - urb persp

\sframe{Part I - Introduction}{

\begin{itemize}
	\item \cite{batty2013new} one view on urban systems; integrative approaches see \cite{lobo2020urban}
	\item Link and distinction between theories and models? see \cite{livet2010ontology} \cite{raimbault2017applied}
	\item Should urban evolution encompass ``everything urban''? $\rightarrow$ examples on the opposite of evolutionary processes?
	\item On the need for interdisciplinarity and multiple perspectives, totally agree: see \cite{pumain2020conclusion}
\end{itemize}


}

\subsubsection{Literature review}

% Urban ecology : // industrial symbiosis?
% Complex adaptive systems: indeed morphogenesis models are not models of urban evolution - confusion in concepts.
% Scaling: allometric growth: do not cite Chen - ! diff in time / static - Q of ergodicity

% bio -> social: units become themselves more complex, heterogenous, fuzzy.

\sframe{Part I - Literature review}{

\begin{itemize}
	\item On urban ecology, see also industrial symbiosis: analogy or equivalence? \cite{raimbault2020spatial}
	\item On Complex Adaptive Systems approach: indeed morphogenesis models are not evolution models - confusion of concepts in the literature? (as in evolutionary economic geography in general? \cite{schamp2010notion})
	\item On Scaling Theories: allometric growth/scaling is linked to scaling but not the same concept - \cite{chen2009analytical} \cite{lang2019reinvestigating}; issue of non-ergodicity \cite{pumain2012multi}
	\item When switching from biological to social evolution, evolution units become more complex, heterogenous, fuzzy.
\end{itemize}


}


\subsubsection{Model outline}

% disagreements analogy or more universal process? -> importance of understanding epistemo aspects of transfers of concepts / universal processes (structural analogy of models/theories?) / nature of interdisciplinarity
% p16: what is a sociocultural entity is fuzzy (at this stage?)

% Proposing physical entities as elementary units is challenging:
%.  - link form and function?
%.  - indeed encode information on how to best organize space, but multi-obj optim by multiple agents (so could not be "simple results of selection"?) -> OK p19 "rational without necessarily being optimized to the interests of any human actors"
%.  - def of elementary unit, where to draw the boundary in terms of nature and delimitation - how to define the system - scales?
%.  - functions / informational processes (cit info in complex systems) may be missed? -> ok function ~ encoded: "groups and activities towards which they are oriented" "a way of physically organizing space for some sets of activities and groups" "Be made out of this stuff arranged in this way, for doing these things for groups of people like this"
%   - Q: def of "socially organized"?
%  - Q of scale: micro level of representation only? - idea of a "molar unit": not sure it is useful/necessary?
%  - levels of interaction and strong emergence "After all, people do not interact with such abstract entities" - they do: admin entities, state, etc.


\sframe{Part I - Model outline}{

\begin{itemize}
	\item Evolution as an analogy or as underlying general processes? $\rightarrow$ transfer of concepts/theories/models between disciplines - nature of interdisciplinarity
	\item What exactly is a ``socio-cultural entity'' is still fuzzy at this stage
\end{itemize}

Proposing ``physical'' urban entities is challenging on several points:

\begin{itemize}
	\item Link form-function? (included in definition of formeme)
	\item Selected as optimizing spatial organization - but fitness functions in social/urban environments are tied to multiple actors at multiple scales, with multiple objectives	
	\item How to draw the boundary for units (nature, scale, delimitation), how to define the system
	\item Information essential in complex systems (included in the formeme) \cite{haken2006information}
	\item Why necessarily a ``molar unit''? Can it span multiple scales?
	\item Compatibility with weak/strong emergence? (people interact with higher level entities: state, municipal council) \cite{bedau2000open}
\end{itemize}




}


\subsubsection{Conclusion}

% origin of cities? yes and no -> cf ALife strongly interested in origins of life - these must be evolutionary (prebiotic theories: chemistry evolution) (see comment on Walker, transitions/information)
% reductionism/determinism: totally agree (complexity of evolution)
% not "teleological": should always be for some agents/utility function? (cf Monod \cite{monod2014hasard}
% from the bottom-up - daily life: agree
% mimicry of bio evol? totally agree - cf comment on transfer of concepts between disciplines
% not anti planning: complexity of managing/ designing complex systems from the bottom up: cf field of morphogenetic engineering \cite{doursat2012morphogenetic}

\sframe{Part I: conclusion}{

\begin{itemize}
	\item Origin of cities: ALife strongly interest in Origins of Life, these must have been evolutionary (from chemical evolution to biological evolution): emergence of life as a phase transition \cite{cronin2016beyond}
	\item Reductionism and determinism: totally agree: complexity of evolution, order out of chaos, frozen accidents
	\item Not ``teleological'' but teleonomical: always some agents with some objective functions \cite{monod2014hasard}
	\item From the bottom-up - daily urban life: importance of emergence
	\item Not mimic biological evolution: see comment on transfer of concepts between disciplines
	\item Not anti-planning: complexity of managing/designing complex systems from the bottom-up: see the field of morphogenetic engineering \cite{doursat2012morphogenetic}
\end{itemize}

}



\subsection{Part II}


% signal -> signals and boundaries

\sframe{Part II - Model}{

\begin{itemize}
	\item Difference between urban genome and phenotype? How does the genome express itself? The genome is defined as a set of formeme, which are already ``expressed'' through the function $f$ to define uses. Is the expression the link between $U$ and $H$? Is phenotype the signature?
	\item The sets $P, A, G$ are fixed - does it mean that a given instantiation of the model is not able to endogenously introduce new categories, and thus not able to exhibit open-ended evolution? \cite{taylor2016open}
	\item Tying the genome to a world is a good idea e.g. to account for non-ergodicity, explore ``what-if'' worlds (``Cities as they could be'' \cite{raimbault2020cities})
	\item The genome is somehow independent of function $H$ and information propagation $S$ - could we include these in the genome, or is it a fundamental divergence point with biological evolution?
\end{itemize}


}

\sframe{Part II - Characteristics, similarity, trajectories}{

\begin{itemize}
	\item To what extent already specifying precise signal characteristics corresponds to model implementation - could it stay generic?
	\item Defining distances between genomes is indeed crucial for phylogenetic studies; estimating distances between organisms (based on proteins alignement for example) is a very difficult problem in biology; would the same difficulties transfer to urban evolution, or would some known historical data/spatial models help directly construct the phylogenies? (section 6 defines the tree in a way).
	\item The link between genome distance and spatial distance is crucial in specifying the model of urban evolution (as for example in \cite{raimbault2020model})
	\item To what extent are activity costs and recoding dependent on context and not only on the formeme?
\end{itemize}


}

\subsection{Part III}

\sframe{Part III}{

\begin{itemize}
	\item Innovations in formeme are physical innovations - to what extent does the model capture the evolution of concepts/ideas/knowledge? It may have very different dynamics.
	\item Environmental variations are a way to capture spatial interactions/diffusion through signals.
	\item Spatial interactions are also crucial in the different selection processes detailed.
	\item On the concept of niche, what is the link between spatial niche (as it is used here) and evolutionary niche (local fitness optimum)?
\end{itemize}


}


\sframe{Part IV}{

\begin{itemize}
	\item Could 6 years be too short to observe meaningful evolutionary processes?
	\item Are numerical values for distances directly interpretable?
	\item Is it possible to link longitudinal distance iwth transversal distance? (Extrapolate some divergence time?)
\end{itemize}


}



\section{Discussion}


\sframe{Successive evolutions and complexity}{

% level of complexity/evolution? - successive evolutions - necessaries? // talk ALife Krakauer https://scholar.google.com/citations?user=R-K-FOwAAAAJ&hl=fr&oi=ao : ? - inevitability of life / intelligence
% \cite{cronin2016beyond} Sara Walker  life as a phase transition, life/non-life (biosignatures)
% "without an understanding of what life is, how can we approach understanding its origins" \cite{walker2017re}

Emergence of life as a succession of phase transitions in an information propagation process \cite{cronin2016beyond}

\medskip

``\textit{Without an understanding of what life is, how can we approach understanding its origins}'' \cite{walker2017re}

\medskip

$\rightarrow$ Succession of chemical, biological, cultural evolution, with increasing complexities

\medskip

Should urban evolution be considered as a new phase transition? (see \cite{batty2018inventing}) What is its main support then? (autocatalytic sets, intelligence, ?)


}

\sframe{Epistemology of concept transfer}{

% biological metaphors / transfer of concepts: ex symbiosis, morphogenesis: epistemo work on transfer of concepts?

Differentiation between using metaphors/analogies, transferring concepts and models, and unveiling universal models and theories across disciplines

\medskip

$\rightarrow$ Building a theory of urban evolution should be related to an epistemological work on the nature of interdisciplinarity, the transfer of concepts.

\medskip

$\rightarrow$ Link with reflexivity and complexity itself \cite{raimbault2020relating}

}

\sframe{Morphogenesis, form and function}{

% Ambitious new way of revisiting link between form and function: // disc morphogenesis

A core element of the proposed model is the link between form and function (formeme including form and usages)

\medskip

$\rightarrow$ Revisit the concept of urban morphogenesis? \cite{batty2009centenary} Endogenize this link within the model? More general concept than biological morphogenesis?


}

\sframe{Autopoiesis: system boundaries?}{

% spatio-temporal aspect of evolution / co-evolution niches (cit. Holland?)

% autopoiesis: Q of system def and boundaries

Question of system definition and system boundaries is not very clear in the proposed model (beyond taking into account geographical areas $c$)

\medskip

System boundaries are crucial in evolution / co-evolution (evolution occurs in spatio-temporal domains with isolation): notion of niche

\medskip

$\rightarrow$ Link with the concept of autopoiesis? \textit{Network of processes self-sustaining itself} \cite{bourgine2004autopoiesis} Are urban systems autopoietic?


}

\sframe{Open-ended evolution}{

The sets used to define formemes are fixed in the model description: does the model apply to very long time scales, across multiple time scales, spatial scales (minimal resolution in mentioned in II, what about maximal resolution?), ontological scales?

\medskip

$\rightarrow$ Capacity of the model to endogeneize the emergence of new elements, new innovations, to capture open-ended evolution? \cite{taylor2016open}

}

% multidimensionality? -> ok encoded in formeme?








%%%%%%%%%%%%%%%%%%%%%
\begin{frame}[allowframebreaks]
\frametitle{References}
\bibliographystyle{apalike}
\bibliography{biblio}
\end{frame}
%%%%%%%%%%%%%%%%%%%%%%%%%%%%










\end{document}
