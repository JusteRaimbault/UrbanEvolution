\documentclass[letterpaper]{article}

\usepackage{natbib,alifeconf}  %% The order is important


% *****************
%  Requirements:
% *****************
%
% - All pages sized consistently at 8.5 x 11 inches (US letter size).
% - PDF length <= 8 pages for full papers, <=2 pages for extended
%    abstracts (not including citations).
% - Abstract length <= 250 words.
% - No visible crop marks.
% - Images at no greater than 300 dpi, scaled at 100%.
% - Embedded open type fonts only.
% - All layers flattened.
% - No attachments.
% - All desired links active in the files.

% Note that the PDF file must not exceed 5 MB if it is to be indexed
% by Google Scholar. Additional information about Google Scholar
% can be found here:
% http://www.google.com/intl/en/scholar/inclusion.html.


% If your system does not generate letter format documents by default,
% you can use the following workflow:
% latex example
% bibtex example
% latex example ; latex example
% dvips -o example.ps -t letterSize example.dvi
% ps2pdf example.ps example.pdf


% For pdflatex users:
% The alifeconf style file loads the "graphicx" package, and
% this may lead some users of pdflatex to experience problems.
% These can be fixed by editing the alifeconf.sty file to specify:
% \usepackage[pdftex]{graphicx}
%   instead of
% \usepackage{graphicx}.
% The PDF output generated by pdflatex should match the required
% specifications and obviously the dvips and ps2pdf steps become
% unnecessary.


% Note:  Some laser printers have a serious problem printing TeX
% output. The use of ps type I fonts should avoid this problem.


\title{Morphogenesis, evolution and co-evolution of cities}
\author{Juste Raimbault$^{1,2,3}$\\
\mbox{}\\
$^1$Center for Advanced Spatial Analysis, University College London, London, UK\\
$^2$UPS CNRS 3611 ISC-PIF, Paris, France\\
$^3$UMR CNRS 8504 G{\'e}ographie-cit{\'e}s, Paris, France\\
juste.raimbault@polytechnique.edu} % email of corresponding author

% For several authors from the same institution use the same number to
% refer to one address.
%
% If the names do not fit well on one line use
%         Author 1, Author 2 ... \\ {\Large\bf Author n} ...\\ ...
%
% If the title and author information do not fit in the area
% allocated, place \setlength\titlebox{<new height>} after the
% \documentclass line where <new height> is 2.25in



\begin{document}
\maketitle

\begin{abstract}
% Abstract length should not exceed 250 words
  Urban systems have a complexity in their own but still have been described and modelled using analogies and models imported from other disciplines. In particular, biological metaphors have been widely used in urban planning and design. We argue that fruitful transfer of concepts between biology and urban science can be achieved within the Artificial Life interdisciplinary framework. We illustrate this idea by synthesising a recent stream of research focusing on urban morphogenesis, urban evolution and urban co-evolution. In each case, novel definitions specific to the urban case were developed, while keeping the most proximity to the original disciplines. In that context, we suggest that many more interdisciplinary bridges could potentially be constructed through such disciplinary transfers going beyond purely methodological ones.
\end{abstract}

% keywords
% Urban morphogenesis; Urban evolution; Co-evolution; Interdisciplinarity

%
%-------------------------  METAREVIEW  ------------------------
%This paper describes the evolutionary process of whole cities and sees this from a complex morphogenetic and from a co-evolutionary perspective. All three reviewers pinpoint the same weakness: it is  mainly a description and summary of the author's own work in the past without striking and novel conclusions that are drawn. Reviewer #1 criticizes that the literature work is mainly self-citations, what is consistent with reviewer #3, who criticizes that relevant other works are not discussed at all. This is also supported by the low score given by reviewer #2 in this aspect. In addition, reviewer #2 criticizes that no/little novelty is involved in the overall submission, what is also indicated by the low scores that the other reviewers give for this aspect. As none of the three reviewers suggested acceptance, I conclude that this paper should not be accepted for the conference.
%
%
%
%----------------------- REVIEW 1 ---------------------
%SUBMISSION: 124
%TITLE: Morphogenesis, evolution and co-evolution of cities
%AUTHORS: Juste Raimbault
%
%----------- Overall evaluation -----------
%SCORE: 0 (borderline paper)
%----- TEXT:
%Summary:
%This paper proposes integrating Alife into urban science, providing examples in urban morphogenesis, urban evolution and urban co-evolution. It suggests that many more inter-disciplinary bridges could be constructed through disciplinary transfers going beyond purely methodological ones.
%
%Strengths:
%* The paper proposes interesting cross-disciplinary ideas that will make good discussion for the conference.
%
%Weaknesses:
%* There is an excessive number of self-citations in the paper. It is only a 2 page paper, 15 references are the author's papers. This should be greatly reduced.
%----------- Novelty/Originality -----------
%SCORE: 3 (Some new ideas or new tests of existing theory)
%----------- Writing Clarity -----------
%SCORE: 4 (Clearly written and easy to understand)
%----------- Thoroughness of Literature Review -----------
%SCORE: 3 (Adequate description of related work, with only small gaps)
%----------- Thoroughness of Methods -----------
%SCORE: 1 (No methods provided)
%----------- Relevance to Artificial Life Conference -----------
%SCORE: 3 (Topic can be considered Artificial Life or of interest to some ALife researchers)
%----------- Overall Quality of Work -----------
%SCORE: 3 (Reasonable quality, sufficient for acceptance)
%
%
%
%----------------------- REVIEW 2 ---------------------
%SUBMISSION: 124
%TITLE: Morphogenesis, evolution and co-evolution of cities
%AUTHORS: Juste Raimbault
%
%----------- Overall evaluation -----------
%SCORE: -2 (reject)
%----- TEXT:
%The topic of this paper is of great interest and relevant to many research communities at present, however sadly this paper does not deliver anything beyond a very high-level overview of the author's own work in the area with no real depth, no proper algorithm/method descriptions, results, comparisons of techniques, and no review of the many other related works. The purpose of scientific publishing is to disseminate the results of new scientific research, whether that is a new method, a new analysis, or new insights - this paper sadly does not achieve this.
%----------- Novelty/Originality -----------
%SCORE: 1 (Rehash of work that has been done before)
%----------- Writing Clarity -----------
%SCORE: 2 (Confusing in many places, needs editing throughout)
%----------- Thoroughness of Literature Review -----------
%SCORE: 2 (Some relevant work described, but large gaps in literature review.)
%----------- Thoroughness of Methods -----------
%SCORE: 2 (Gaps in methods; would be impossible to replicate work without additional details)
%----------- Relevance to Artificial Life Conference -----------
%SCORE: 2 (Topic only tangentially related to Artificial Life and better suited to other outlets)
%----------- Overall Quality of Work -----------
%SCORE: 2 (Poor quality, probably reject)
%
%
%
%----------------------- REVIEW 3 ---------------------
%SUBMISSION: 124
%TITLE: Morphogenesis, evolution and co-evolution of cities
%AUTHORS: Juste Raimbault
%
%----------- Overall evaluation -----------
%SCORE: -1 (weak reject)
%----- TEXT:
%This article is about the evolution of cities. It describes several possibilities and evolutions ranging from the generation of the road structure to a real co-evolution of urban systems. It is more an incomplete history than a real contribution. For example, it omits work using parameterised grammar generating systems.
%----------- Novelty/Originality -----------
%SCORE: 2 (Incremental work that advances existing ideas)
%----------- Writing Clarity -----------
%SCORE: 3 (Mostly well written, but a few confusing portions (described in review))
%----------- Thoroughness of Literature Review -----------
%SCORE: 3 (Adequate description of related work, with only small gaps)
%----------- Thoroughness of Methods -----------
%SCORE: 1 (No methods provided)
%----------- Relevance to Artificial Life Conference -----------
%SCORE: 4 (Topic is clearly relevant to Artificial Life community)
%----------- Overall Quality of Work -----------
%SCORE: 2 (Poor quality, probably reject)
%


\section{Introduction}

The emergence of cities is tightly linked to the complexification of societies, and can be understood as one further step in cultural evolution \citep{sjoberg1965origin}. To what extent this additional layer of complexity lead to urban systems that themselves evolve with proper rules and processes remains an open question. Properties such as scaling which can be exhibited across successive level of emergence, from biological to social and urban systems \citep{youn2018scaling}, suggest at least some analogies between these different scales. In that context, biological metaphors have been used to model, understand, and design cities \citep{batty2009centenary}. Several disciplines have developed theories and models to understand the dynamics of cities \citep{pumain2020conclusion}, among which Artificial Life approaches have shown some relevance.

\cite{gershenson2013living} suggests that the application of concepts imported from the study of living systems can bring solutions to urban issues, such as traffic congestion, logistics, or more generally sustainability. Methods such as fractals can be used to quantify urban form \citep{chen2010modeling}. Urban traffic can also be modelled using ALife methods \citep{yoshioka2017macroscopic}. \cite{raimbault2020cities} shows a citation map of the scientific landscape loosely related to ALife in the study of urban systems, ranging from cellular automata models of urban growth to evolutionary computing applied to urban design or optimisation.


Beyond using similar methodologies or techniques, what is a common feature of the study of complex systems, the application of biological metaphors to the study of cities is one crucial feature in which ALife approaches have contributed to urban science, such as urban morphogenesis. The field of urban ecology, and more particularly industrial symbiosis \citep{chertow2007uncovering}, in which companies are understood as ecosystems, is one other example.

The purpose of this contribution is to illustrate that effective transfer of concepts, which go beyond metaphors with new definitions and an adaptation of concepts, can occur between biology and urban science, and be fruitful to understand the dynamics of urban systems. We furthermore claim that the ALife interdisciplinary framework is a privileged scientific background to operate such transfers. To do so, we propose to synthesise a recent stream of research bridging ALife and urban science, which originated with the study of co-evolutionary processes in urban systems \citep{raimbault2018caracterisation}. In that context, different concepts were transferred at different scales: urban morphogenesis at the scale of the city itself, and urban evolution and co-evolution at the scale of the system of cities.



\section{Urban morphogenesis}

Urban morphogenesis can be in it simplest sense understood as the growth of urban form, at the scale of a city or an urban area. \cite{raimbault2014hybrid} introduce for example a cellular automaton coupled with a dynamical road network, which produces a variety of dynamical regimes of urban growth \citep{raimbault2017identification}. \cite{raimbault2019generating} compare and calibrate generative models at the district scale. \cite{raimbault2018multi} show the complementarity of multiple heuristics to simulate the morphogenesis of transportation networks. The concept of urban morphogenesis can also be made closer to its biological counterpart by showing how reaction-diffusion models can accurately reproduce existing urban forms, as shown by \cite{raimbault2018calibration} for urban areas in Europe. Finally, by constructing a definition of urban morphogenesis as \emph{the strong coupling between the growth of urban form and the emergence of urban functions}, \cite{raimbault2018caracterisation} introduces an original viewpoint in which the transfer from biology is crucial. Following this positioning, the morphogenesis model studied by \cite{raimbault2019urban} uses transportation network to integrate a functional aspect into the dynamics of urban growth. Processes of urban morphogenesis can be seen as occurring within territorial niches \citep{holland2012signals}, what suggests the relevance of studying urban evolution within and between these niches at larger scales.

\section{Urban evolution}

The idea of evolving entities within urban systems directly evokes theories of cultural evolution \citep{mesoudi2016cultural}, as social and cultural processes are occurring within cities, which are furthermore highlighted as incubators of social change \citep{pumain2019complexity}. The concept of urban evolution however goes beyond this, as cities themselves can be seen as evolving entities. \cite{pumain1997pour} proposed an \emph{evolutionary theory of cities} to understand the dynamics of urban systems, in the sense of adaptive complex systems. \cite{pumain2017urban} synthesise the empirical and modeling results which were obtained in the direct heritage of this positioning.

Approaches of urban evolution with a definition closer to its biological and social counterparts however remains to be constructed. A partial proposition for a model of urban evolution is done by \cite{raimbault2020model}. Extending the model of \cite{favaro2011gibrat}, the main ingredients are innovations which propagate between cities. Therein, transmission processes are captured through diffusion, while transformation processes occur as mutations when new innovations are created. A one-dimensional urban genome is thus the adoption of each innovation in each city. While this model remains stylised and limited, first in the fact that other urban dimensions are not accounted for, secondly as innovations all compete on the same dimension, it however provides a definition and operationalisation of urban evolution which includes the formal processes required for evolution (transmission, transformation, isolation of sub-systems for differentiation) \citep{durham1991coevolution}.


\section{Co-evolution in urban systems}


Finally, one concept for which a relevant transfer can be achieved between biology and urban science is the concept of co-evolution. \cite{raimbault2018caracterisation} introduces a multi-level definition, with at the intermediate level the possibility of \emph{statistically co-evolving population of urban entities within territorial niches}. More concretely, it is applied to the co-evolution between transportation networks of cities, and hypothesises that a circular relation between some properties of cities and some of transportation networks can be observed within some geographical regions. A method to characterise it on spatio-temporal data was introduced by \cite{raimbault2017identification}. At the mesoscopic scale of urban areas, the morphogenesis model of \cite{raimbault2019generating} effectively captures such a co-evolution between indicators of urban morphology and the topology of road networks. \cite{le2015modeling} propose a model strongly coupling land-use dynamics with transportation infrastructure dynamics, integrating network governance agents. At the macroscopic scale of the system of cities, \cite{raimbault2018modeling} study a co-evolution model between cities and inter-city networks, and show that diverse regimes of co-evolution can be produced. \cite{raimbault2020hierarchy} extends this model with a more precise representation of transportation networks. These different applications show how this transfer of concept is fruitful to the understanding of urban dynamics.


\section{Discussion}


Several further transfers and developments can be suggested following this synthesis. A main research direction also strongly inspired by ALife and complexity is the construction of multi-scale models of urban evolution and urban dynamics. Previous efforts have remained limited as they did not used distinct ontologies and strong coupling between scales. For example, \cite{batty2005agents} introduces a common formalisation using cellular automata to simulate urban dynamics from the neighbourhood scale to the system of cities scale. \cite{murcio2015urban} study a model simulating urban migrations migrations at different spatial ranges. A first step towards such models has been recently proposed by \cite{raimbault2021strong}.


Such multi-scale models of urban dynamics would be essential tools for a sustainable territorial planning \citep{rozenblat2018conclusion}. Similarly, the benchmarking and comparison of multiple models for urban dynamics, as illustrated by \cite{raimbault2020empowering} for multiple urban systems worldwide, is a crucial step to ensure the robustness of policies.

Finally, many concepts originating from biology and studied across disciplines in the field of Artificial Life, remain relevant candidates for a strong transfer of concept and an operationalisation within urban science. For example, biomimicry has been put forward as an effective tool for urban design and architecture \citep{taylor2017art}. The concept of autopoiesis \citep{bourgine2004autopoiesis}, which is tightly linked with morphogenesis but also cognition, remains to be investigated in the case of urban systems. Urban computing and collective intelligence, as research fields such as smart cities and digital twins are booming \citep{batty2018digital}, may be linked to evolutionary computation, and more generally to studies of collective computation in biological systems. Altogether, we suggest that the integration of ALife into urban science and underlying transfer of concepts between disciplines, are relevant research directions for both.




%\section{Acknowledgements}
%


\footnotesize
\bibliographystyle{apalike}
\bibliography{biblio} % replace by the name of your .bib file


\end{document}


%\begin{figure}[t]
%\begin{center}
%\includegraphics{}
%\caption{}
%\label{fig1}
%\end{center}
%\end{figure}



%\begin{table}[h]
%\center{
%\begin{tabular}{|c|c|c|c|}\hline
%Name & Result & Bonus $b_i$ & Difficulty\\ \hline\hline
%Echo & I/O   & 1 & --\\
%Equals &$\neg(A\ {\rm xor}\ B)$&6& 4 \\ \hline
%\end{tabular}
%}
%\vskip 0.25cm
%\caption{}
%\end{table}

