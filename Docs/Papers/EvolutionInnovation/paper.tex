\documentclass[letterpaper]{article}

\usepackage{natbib,alifeconf}  %% The order is important


% *****************
%  Requirements:
% *****************
%
% - All pages sized consistently at 8.5 x 11 inches (US letter size).
% - PDF length <= 8 pages for full papers, <=2 pages for extended
%    abstracts (not including citations).
% - Abstract length <= 250 words.
% - No visible crop marks.
% - Images at no greater than 300 dpi, scaled at 100%.
% - Embedded open type fonts only.
% - All layers flattened.
% - No attachments.
% - All desired links active in the files.

% Note that the PDF file must not exceed 5 MB if it is to be indexed
% by Google Scholar. Additional information about Google Scholar
% can be found here:
% http://www.google.com/intl/en/scholar/inclusion.html.


% If your system does not generate letter format documents by default,
% you can use the following workflow:
% latex example
% bibtex example
% latex example ; latex example
% dvips -o example.ps -t letterSize example.dvi
% ps2pdf example.ps example.pdf


% For pdflatex users:
% The alifeconf style file loads the "graphicx" package, and
% this may lead some users of pdflatex to experience problems.
% These can be fixed by editing the alifeconf.sty file to specify:
% \usepackage[pdftex]{graphicx}
%   instead of
% \usepackage{graphicx}.
% The PDF output generated by pdflatex should match the required
% specifications and obviously the dvips and ps2pdf steps become
% unnecessary.


% Note:  Some laser printers have a serious problem printing TeX
% output. The use of ps type I fonts should avoid this problem.


\title{A simple model of urban evolution based on innovation diffusion}
\author{Juste Raimbault$^{1,2,3}$\\
\mbox{}\\
$^1$Center for Advanced Spatial Analysis, University College London\\
$^2$UPS CNRS 3611 ISC-PIF\\
$^3$UMR CNRS 8504 G{\'e}ographie-cit{\'e}s\\
juste.raimbault@polytechnique.edu} % email of corresponding author

% For several authors from the same institution use the same number to
% refer to one address.
%
% If the names do not fit well on one line use
%         Author 1, Author 2 ... \\ {\Large\bf Author n} ...\\ ...
%
% If the title and author information do not fit in the area
% allocated, place \setlength\titlebox{<new height>} after the
% \documentclass line where <new height> is 2.25in



\begin{document}
\maketitle

\begin{abstract}
% Abstract length should not exceed 250 words
  
\end{abstract}

\section{Introduction}

% https://scholar.google.com/scholar?hl=fr&as_sdt=0%2C5&q=meme+gene+coevolution&btnG=


\cite{votsis2019urban}

\section{Urban evolution model}

\subsection{Rationale}

The principal idea in the model is to include a concept of ``Urban DNA'', i.e. a kind of genome that cities would be exchanging and which would undergo mutation processes. A suitable candidate is to build on the concept of \emph{meme} introduced in the field of cultural evolution. However, several particularities must be stressed out when working with urban systems.

\cite{d2014urban}

\subsection{Model description}

\cite{favaro2011gibrat}


\section{Results}


\section{Discussion}


%\section{Acknowledgements}


\footnotesize
\bibliographystyle{apalike}
\bibliography{biblio} % replace by the name of your .bib file


\end{document}



%\begin{figure}[t]
%\begin{center}
%\includegraphics[width=2.1in,angle=-90]{fig1.eps}
%\caption{``Energies'' (inferiorities) of strings in a first-order
%  phase transition with latent heat $\Delta\epsilon$.}
%\label{fig1}
%\end{center}
%\end{figure}


%
%\begin{table}[h]
%\center{
%\begin{tabular}{|c|c|c|c|}\hline
%Name & Result & Bonus $b_i$ & Difficulty\\ \hline\hline
%Echo & I/O   & 1 & --\\
%Not  & $\neg A$ & 2 & 1 \\
%Nand & $\neg(A\wedge B)$ & 2 & 1 \\
%Not Or & $\neg A \vee B$ & 3 & 2 \\
%And  &  $ A \wedge B $   & 3 & 2 \\
%Or   &  $ A \vee B $     & 4 & 3 \\
%And Not & $A\wedge\neg B$& 4 & 3 \\
%Nor  & $\neg(A\vee B)$   & 5 & 4 \\
%Xor  & $ A\ {\rm xor}\ B$ &   6 & 4 \\
%Equals &$\neg(A\ {\rm xor}\ B)$&6& 4 \\ \hline
%\end{tabular}
%}
%\vskip 0.25cm
%\caption{Logical calculations on random inputs $A$ and $B$ rewarded,
%bonuses, and difficulty (in minimum number of {\tt nand} instructions
%required). Bonuses $b_i$ increase the speed of a CPU by a factor
%$\nu_i=1+2^{b_i-3}$.}
%\end{table}
%


