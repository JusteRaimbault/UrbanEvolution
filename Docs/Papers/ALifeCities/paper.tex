\documentclass[letterpaper]{article}

\usepackage{natbib,alifeconf}  %% The order is important


% *****************
%  Requirements:
% *****************
%
% - All pages sized consistently at 8.5 x 11 inches (US letter size).
% - PDF length <= 8 pages for full papers, <=2 pages for extended
%    abstracts (not including citations).
% - Abstract length <= 250 words.
% - No visible crop marks.
% - Images at no greater than 300 dpi, scaled at 100%.
% - Embedded open type fonts only.
% - All layers flattened.
% - No attachments.
% - All desired links active in the files.

% Note that the PDF file must not exceed 5 MB if it is to be indexed
% by Google Scholar. Additional information about Google Scholar
% can be found here:
% http://www.google.com/intl/en/scholar/inclusion.html.


% If your system does not generate letter format documents by default,
% you can use the following workflow:
% latex example
% bibtex example
% latex example ; latex example
% dvips -o example.ps -t letterSize example.dvi
% ps2pdf example.ps example.pdf


% For pdflatex users:
% The alifeconf style file loads the "graphicx" package, and
% this may lead some users of pdflatex to experience problems.
% These can be fixed by editing the alifeconf.sty file to specify:
% \usepackage[pdftex]{graphicx}
%   instead of
% \usepackage{graphicx}.
% The PDF output generated by pdflatex should match the required
% specifications and obviously the dvips and ps2pdf steps become
% unnecessary.


% Note:  Some laser printers have a serious problem printing TeX
% output. The use of ps type I fonts should avoid this problem.


\title{Cities as they could be: Artificial Life and Urban Systems}
\author{Juste Raimbault$^{1,2,3}$\\
\mbox{}\\
$^1$Center for Advanced Spatial Analysis, University College London\\
$^2$UPS CNRS 3611 ISC-PIF\\
$^3$UMR CNRS 8504 G{\'e}ographie-cit{\'e}s\\
juste.raimbault@polytechnique.edu} % email of corresponding author

% For several authors from the same institution use the same number to
% refer to one address.
%
% If the names do not fit well on one line use
%         Author 1, Author 2 ... \\ {\Large\bf Author n} ...\\ ...
%
% If the title and author information do not fit in the area
% allocated, place \setlength\titlebox{<new height>} after the
% \documentclass line where <new height> is 2.25in



\begin{document}
\maketitle

\begin{abstract}
% Abstract length should not exceed 250 words
  
\end{abstract}

% \cite{pattee1995artificial} epistemic cut: simulation / hard or wet ALife / nature of computation % not relevant here?


\section{Introduction}


The understanding of processes driving the growth of cities, or more generally the evolution of urban systems, and approaches for a sustainable design and management of these, have for more than 100 years been tightly linked to concepts borrowed from biology such as evolution \citep{batty2009centenary}. Inventing novel ways to design cities, beyond predicting their evolution \citep{batty2018inventing}, is a major asset to tackle climate change and sustainability issues \citep{iepsfc2018science}. In that context, Artificial Life (ALife) approaches to urban systems have several advantages, including knowledge transfer from biology and ecology where relevant concepts such as resilience or morphogenesis are thoroughly studied, a strong practice of interdisciplinarity, or methodologies and tools such as agent-based modeling and cellular automata, among others. Following the seminal view of ``Life as it could be'' by \cite{langton1986studying}, an explicit ALife take on urban issues would consist in studying ``Cities as they could be''. Although not explicitly listed in application domains of ALife by \cite{kim2006comprehensive}, is is included through the use of methods and in relation with economic models. Two of archetypal ``would-be'' worlds of \cite{casti1997would} are related to urban systems (transportation and resource exploitation). \cite{doi:10.1162/isala00135} recently introduced a conceptual frame considering cities from an organismic perspective, while \cite{doi:10.1162/isala00134} showed that open-endedness has a strong potential to develop sustainable social systems.


Diverse streams of research on urban systems have already linked with ALife. From a methodological viewpoint, the use of cellular automata and agent-based models for urban growth and urban dynamics has a long history \cite{torrens2003automata}. These can be used for example to generate building layouts and road networks of synthetic cities \citep{kato1998alife} or at the scale of districts \citep{raimbault2019generating}. Integrating such models into evolutionary computation algorithms widens the scope of possible synthetic cities \citep{kato2000modeling}. Other dimensions such as land prices and residential dynamics can be grasped using agent-based models \citep{takizawa2000simulation}. Generative processes can also be used for interactive urban design \citep{openshaw1995developing}. The study of urban morphology may be done with methods used to study morphogenesis, and \cite{medda2009morphogenetic} apply reaction-diffusion equation to model the relation between transportation and land-use, while \cite{raimbault2018calibration} shows that the combination of aggregation and diffusion produces realistic urban forms. \cite{d2013simulating} explores possible future sustainable urban morphologies. \cite{lucic2002transportation} use bio-inspired algorithms to solve difficult transportation planning problems. The concept of urban metabolism introduced by \cite{olsen1982urban} also comes as a transfer from biology.


Moreover, the field of Artificial Intelligence (AI) has also numerous application related to urban systems. \cite{wu2010artificial} review AI application to the prediction of land-use dynamics, unveiling a very broad range of methods ranging from evolutionary computation to neural networks, and suggesting that integrative and interdisciplinary approaches still lack for more robust urban applications. \cite{white1989artificial} show for example that a neural network trained appropriately can learn to plan transportation infrastructures. \cite{zheng2014urban} define the emerging field of Urban Computing as the convergence of ubiquitous urban data with artificial intelligence and new urban services, with varied domains of application including transportation, economy, environment, and planning. AI can for example be applied in real-time conditions to manage highway traffic \citep{ma2009real}. Other urban dimensions which require accurate predictions with a high spatio-temporal resolution, such as water demand \citep{adamowski2010comparison} or electric vehicles grid management \citep{rigas2014managing}, are other examples where AI is successfully applied.


Thus, as both ALife and AI have been broadly applied to urban systems, we can first ask what are their respective extent in terms of methodologies, tools, concepts, and application domains, and secondly what are their remaining potentialities to enhance the understanding and management of cities, in other words what research directions and concepts in that particular context remain to be explored. This paper proposes to tackle these two questions by means of a systematic literature mapping method based on citation networks. More precisely, our contribution (i) constructs and explores a large citation network of around 250,000 papers, to map the respective contributions of ALife and AI to the urban literature, and their relations; and (ii) explores more thoroughly crucial concepts still loosely applied or understood in an urban system perspective.


The rest of this paper is organized as follows: we develop in the next section the bibliometric analysis based on citation networks to map the scope of artificial life and artificial intelligence approaches to urban systems. Building on this systematic mapping, we then review the principal points in which artificial life can significantly inform the study of urban systems. We finally discuss research directions opened by taking such a viewpoint of ``Cities as They Could Be''.


\section{Bibliometric analysis}

\subsection{Method}

% Q: a priori or a posteriori analysis?

Literature mapping and quantitative bibliometrics have been widely used to reinforce knowledge in most disciplines, and are part of a field of study in itself \cite{leydesdorff2001challenge}. They are furthermore important to enhance reflexivity which is crucial in disciplines studying socio-technical systems \cite{raimbault2019empowering}. In the case of artificial intelligence, several mappings have been proposed, for example from a semantic \citep{van1993neural}, spatialized \citep{niu2016global}, or journal-level \citep{ibanez2011using} viewpoint. \cite{squazzoni2013social} analyze the impact of the Journal on Artificial Societies and Social Simulation. \cite{aguilar2014past} show the evolution of theme frequency in time for the Artificial Life journal. There is however to the best of our knowledge no previous attempt of such an exercise for Artificial Life at a large scale. We propose here such a literature mapping approach to both ALife and AI, in the specific context of urban systems applications.

We use therefore a citation network analysis, applying the methods and tools developed by \cite{raimbault2019exploration}. In a nutshell, citation networks are constructed by first constituting a seed corpus using a keyword search, and then by collecting papers citing papers in this corpus, recursively to a certain level.

The keyword we use to constitute our seed corpus follows , and we thus consider only the requests \texttt{Artificial Intelligence AND Urban} for AI approaches and \texttt{Artificial Life AND Urban} for ALife. Using alternative terms such as ``machine learning'' or ``city'' expands the scope too much and yield non relevant results.
% "smart city" ok should come by itself in the scope - less a priori as possible
We construct initial seed corpuses using these requests
% 20 limit already fine for ALife - NO
% -> add by hand refs? such as \cite{kato1998alife}
% AI: start with 50 - seems relevant up to 100 at least but strong intersection with CA approaches.
% Initial: AI 200; ALife 50.
Some references which obviously did not fit the scope were manually removed. 
%sociology / gender studies: WTF?


\subsection{Results}

% - interesting insight: on which communities do AI and ALife intersect (hypothesis: ABM, evolutionary computation?)




\section{Cities as they could be: strengthening ALife concepts in urban science}


We turn now to a more thorough development of some concepts related to ALife which should have either a high theoretical importance for the study of urban systems, or a high potentiality to introduce novel approaches. This list stems from the conjunction of open issues in urban science \citep{lobo2020urban} with conclusions from the previous literature mapping on underexplored paths. It can be understood as main arguments of why ALife concepts may help understanding, planning, designing and managing in a better and more sustainable way.



% Note: wu2010artificial alife subset of AI? ~ contrary in theory.
% Insist here that most approaches are "tools"/technical: what is rare and could really make a difference are strong interdisciplinary works, concepts transfers (with proper defs / shared etc.)
% ! if it is true from the literature review

% - process understanding: "the real smart"


\subsection{Tools and methods}

% - tools: simulation models etc.: already quite mature: which perspectives?



\subsection{Biomimetism}

% - biomimetism (biomimicry?)

\cite{taylor2017art}


\subsection{Urban ecology}

% - urban metabolism


\subsection{Morphogenesis and autopoiesis}

% - morphogenesis / autopoiesis
% form and function: fundamental question. morphogenetic engineering
% autopoiesis: the question of boundaries; sustainibility.





\subsection{Urban evolutionary theories}


% - proper defs of ``Urban dna'', evolutionary processes

The definition of an ``Urban DNA'', i.e. an extension of the gene concept in evolution or of the \emph{meme} concept in cultural evolution, or even an approach combining different types of replicators \cite{bull2000meme}, remains an open question.



\subsection{Innovation in urban systems}

% - innovation diffusion 


\subsection{Linking Urban ALife and AI: urban computing}

% broad issues: computing territorial systems ? cultural evol to urban evol ?



\section{Discussion}

% what remains to be done
% / open problems in "urban evolution"
%
%  - proper definition of urban genotype / phenotype (higher dimension of cultural evol. ?)
%  - def and understanding of co-evolution -> with open ended, in previous section?
%. - understanding of morphogenesis - in the sense of urban form and function
%. - understanding of autopoiesis: urban boundaries is an open Q (<- use open Q as arg why analogy is relevant ?)


%\section{Acknowledgements}


\footnotesize
\bibliographystyle{apalike}
\bibliography{biblio} % replace by the name of your .bib file


\end{document}



%\begin{figure}[t]
%\begin{center}
%\includegraphics[width=2.1in,angle=-90]{fig1.eps}
%\caption{``Energies'' (inferiorities) of strings in a first-order
%  phase transition with latent heat $\Delta\epsilon$.}
%\label{fig1}
%\end{center}
%\end{figure}


%
%\begin{table}[h]
%\center{
%\begin{tabular}{|c|c|c|c|}\hline
%Name & Result & Bonus $b_i$ & Difficulty\\ \hline\hline
%Echo & I/O   & 1 & --\\
%Not  & $\neg A$ & 2 & 1 \\
%Nand & $\neg(A\wedge B)$ & 2 & 1 \\
%Not Or & $\neg A \vee B$ & 3 & 2 \\
%And  &  $ A \wedge B $   & 3 & 2 \\
%Or   &  $ A \vee B $     & 4 & 3 \\
%And Not & $A\wedge\neg B$& 4 & 3 \\
%Nor  & $\neg(A\vee B)$   & 5 & 4 \\
%Xor  & $ A\ {\rm xor}\ B$ &   6 & 4 \\
%Equals &$\neg(A\ {\rm xor}\ B)$&6& 4 \\ \hline
%\end{tabular}
%}
%\vskip 0.25cm
%\caption{Logical calculations on random inputs $A$ and $B$ rewarded,
%bonuses, and difficulty (in minimum number of {\tt nand} instructions
%required). Bonuses $b_i$ increase the speed of a CPU by a factor
%$\nu_i=1+2^{b_i-3}$.}
%\end{table}
%


