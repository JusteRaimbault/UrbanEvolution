\documentclass[letterpaper]{article}

\usepackage{natbib,alifeconf}  %% The order is important


% *****************
%  Requirements:
% *****************
%
% - All pages sized consistently at 8.5 x 11 inches (US letter size).
% - PDF length <= 8 pages for full papers, <=2 pages for extended
%    abstracts (not including citations).
% - Abstract length <= 250 words.
% - No visible crop marks.
% - Images at no greater than 300 dpi, scaled at 100%.
% - Embedded open type fonts only.
% - All layers flattened.
% - No attachments.
% - All desired links active in the files.

% Note that the PDF file must not exceed 5 MB if it is to be indexed
% by Google Scholar. Additional information about Google Scholar
% can be found here:
% http://www.google.com/intl/en/scholar/inclusion.html.


% If your system does not generate letter format documents by default,
% you can use the following workflow:
% latex example
% bibtex example
% latex example ; latex example
% dvips -o example.ps -t letterSize example.dvi
% ps2pdf example.ps example.pdf


% For pdflatex users:
% The alifeconf style file loads the "graphicx" package, and
% this may lead some users of pdflatex to experience problems.
% These can be fixed by editing the alifeconf.sty file to specify:
% \usepackage[pdftex]{graphicx}
%   instead of
% \usepackage{graphicx}.
% The PDF output generated by pdflatex should match the required
% specifications and obviously the dvips and ps2pdf steps become
% unnecessary.


% Note:  Some laser printers have a serious problem printing TeX
% output. The use of ps type I fonts should avoid this problem.


\title{Cities as they could be: Artificial Life and Urban Systems}
\author{Juste Raimbault$^{1,2,3}$\\
\mbox{}\\
$^1$Center for Advanced Spatial Analysis, University College London\\
$^2$UPS CNRS 3611 ISC-PIF\\
$^3$UMR CNRS 8504 G{\'e}ographie-cit{\'e}s
juste.raimbault@polytechnique.edu} % email of corresponding author

% For several authors from the same institution use the same number to
% refer to one address.
%
% If the names do not fit well on one line use
%         Author 1, Author 2 ... \\ {\Large\bf Author n} ...\\ ...
%
% If the title and author information do not fit in the area
% allocated, place \setlength\titlebox{<new height>} after the
% \documentclass line where <new height> is 2.25in



\begin{document}
\maketitle

\begin{abstract}
% Abstract length should not exceed 250 words
  
\end{abstract}

\section{Introduction}


\cite{batty2009centenary} % + relevant in other paper: morphogenesis, urban evolution?

\cite{kim2006comprehensive} applications of alife


\cite{pattee1995artificial} epistemic cut: simulation / hard or wet ALife / nature of computation

% urban DNA ?

\cite{torrens2003automata} CA and ABM

\cite{openshaw1995developing} GIS


% broad issues: computing territorial systems ? cultural evol to urban evol ?


\section{Artificial intelligence}


\cite{wu2010artificial} AI applied to land-use dynamics

\cite{white1989artificial}


\section{Artificial life}

\cite{medda2009morphogenetic} Turing equations

\cite{olsen1982urban} urban metabolism and cycles

\cite{kato1998alife} road network / building layout

\cite{kato2000modeling} CA/GA dynamical

% some procedural modeling ?

\section{Bibliometric analysis}

%%%%
% Method
% - keyword req (separate alife / AI ? - size pb) || initial corpus ?
% - cit nw construction
% Q: a priori or a posteriori analysis?

The keyword request follows \cite{niu2016global} and uses only \texttt{Artificial Intelligence AND Urban} for AI approaches and \texttt{Artificial Life AND Urban} for ALife. Using alternative terms such as ``machine learning'' or ``city'' expands the scope too much and yield non relevant results.
% "smart city" ok should come by itself in the scope - less a priori as possible
We construct initial seed corpuses using these requests
% 20 limit already fine for ALife
% -> add by hand refs? such as \cite{kato1998alife}
% AI: start with 50 - seems relevant up to 100 at least but strong intersection with CA approaches.
%



\section{Cities as they could be}



% How ALife concepts and approaches can help understanding / managing / planning / designing urban systems

% - biomimetism
% - morphogenesis / autopoiesis
% - tools: simulation models etc.
% - urban metabolism
% - process understanding: "the real smart"
% - innovation diffusion ? (! not too close from urb evol th: step back? )




\section{Discussion}

% what remains to be done
% / open problems in "urban evolution"
%
%  - proper definition of urban genotype / phenotype (higher dimension of cultural evol. ?)
%  - def and understanding of co-evolution
%. - understanding of morphogenesis - in the sense of urban form and function
%. - understanding of autopoiesis: urban boundaries is an open Q (<- use open Q as arg why analogy is relevant ?)


%\section{Acknowledgements}


\footnotesize
\bibliographystyle{apalike}
\bibliography{biblio} % replace by the name of your .bib file


\end{document}



%\begin{figure}[t]
%\begin{center}
%\includegraphics[width=2.1in,angle=-90]{fig1.eps}
%\caption{``Energies'' (inferiorities) of strings in a first-order
%  phase transition with latent heat $\Delta\epsilon$.}
%\label{fig1}
%\end{center}
%\end{figure}


%
%\begin{table}[h]
%\center{
%\begin{tabular}{|c|c|c|c|}\hline
%Name & Result & Bonus $b_i$ & Difficulty\\ \hline\hline
%Echo & I/O   & 1 & --\\
%Not  & $\neg A$ & 2 & 1 \\
%Nand & $\neg(A\wedge B)$ & 2 & 1 \\
%Not Or & $\neg A \vee B$ & 3 & 2 \\
%And  &  $ A \wedge B $   & 3 & 2 \\
%Or   &  $ A \vee B $     & 4 & 3 \\
%And Not & $A\wedge\neg B$& 4 & 3 \\
%Nor  & $\neg(A\vee B)$   & 5 & 4 \\
%Xor  & $ A\ {\rm xor}\ B$ &   6 & 4 \\
%Equals &$\neg(A\ {\rm xor}\ B)$&6& 4 \\ \hline
%\end{tabular}
%}
%\vskip 0.25cm
%\caption{Logical calculations on random inputs $A$ and $B$ rewarded,
%bonuses, and difficulty (in minimum number of {\tt nand} instructions
%required). Bonuses $b_i$ increase the speed of a CPU by a factor
%$\nu_i=1+2^{b_i-3}$.}
%\end{table}
%


